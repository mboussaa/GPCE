\section{Evaluation}
So far, we have presented a sound procedure and automated component-based framework for extracting the non-functional properties of generated code. In this section, we evaluate the implementation of our approach by explaining the design of our empirical study and the different methods we used to assess the effectiveness of our approach. 
The experimental material is available for replication purposes\footnote{\url{https://testingcodegenerators.wordpress.com/}}.
\subsection{Experimental Setup}
\subsubsection{Code Generators Under Test: HAXE compilers}
In order to test the applicability of our approach, we conduct experiments on a popular high-level programming language called HAXE and its code generators.

Haxe is an open source toolkit for cross-platform development which compiles to a number of different programming platforms, including JavaScript, Flash, PHP, C++, C\# and Java. Haxe involves many features: the Haxe language, multi-platform compilers, and different native libraries. 
The Haxe language is a high-level programming language which is strictly typed. This language supports both functional programming and object-oriented programming paradigms. It has a common type hierarchy, making certain API available on every targeted platform.
Moreover, Haxe comes with a set of compilers that translate manually-written code (in Haxe language) to different target languages and platforms. 
Haxe code can be compiled for applications running on desktop, mobile and web platforms. Compilers ensure the correctness of user code in terms of syntax and type safety.
Haxe comes also with a set of standard libraries that can be used on all supported targets and platform-specific libraries for each of them.

The process of code transformation and generation can be described as following: Haxe compilers analyzes the source code written in Haxe language then, the code is checked and parsed into a typed structure, resulting in a typed abstract syntax tree (AST). This AST is optimized and transformed afterwards to produce source code for target platform/language.

Haxe offers the option of choosing which platform to target for each program using a command-line tool. Moreover, some optimizations and debugging information can be enabled through CLI but in our experiments, we did not turned on any further options. 

\subsubsection{Cross-platform Benchmark}
One way to prove the effectiveness of our approach is to create benchmarks. Thus, we use the Haxe language and its code generators to build a cross-platform benchmark. The proposed benchmark is composed of a collection of cross-platform libraries that can be compiled to different targets\footnote{\url{http://thx-lib.org/}}. In these experiments, we consider five Haxe code generators to test: Java, JS, C++, CS, and PHP code generator. 

In fact, each Haxe library comes with an API and a set of test suites. These tests, written in Haxe, represent a set of unit tests that covers the different functions of the API. The main task of these tests is to check the correct functional behavior of generated programs once generated code is executed within the target platform. To prepare our benchmark, we have removed all the tests that fail to compile to our five targets (i.e., errors, crashes and failures) and we kept only test suites that are functionally correct. Moreover, we added manually new test cases to some libraries in order to extend the number of test suites. the number of test suites depends on the number of functions within the Haxe library.

We use then, these test suites then, to generate a load and stress the target library. This can be useful to study the impact of this load on the resource usage of the system. For example, if one test suite consumes a lot of resources for a specific target, then this could be explained by the fact that the code generator has produced code that is very greedy in terms of resources.

Thus, we run each test suite 1000 times to get comparable values in terms of resource usage.
Table 2 describes the Haxe libraries that we have selected from this benchmark to evaluate our
approach.

\begin{table}[h]
	\centering

	\begin{tabular}{|c|c|p{4.3cm}|}
		\hline
		\textbf{Library} & \textbf{\#TestSuites} & \textbf{Description} \\
		\hhline{|=|=|=|}
		Color  &  19 &  Color conversion from/to any color space   \\ \hline
		Core & 51  & Provides extensions to many types  \\ \hline
		Hxmath & 6  & A 2D/3D math library  \\ \hline
	    Format  &  4 & Format library such as dates, number formats   \\ \hline
		Promise & 3  & Library for lightweight promises and futures  \\ \hline
		Culture & 4  & Localization library for Haxe \\ \hline
		Math & 3  & Generation of random values \\ \hline
	\end{tabular}
		\caption{Description of selected benchmark libraries}
		\label{my-label}
\end{table}

\subsubsection{Evaluation Metrics Used}
We use to evaluate the efficiency of generated code using the following non-functional metrics:

-\textit{Memory Usage (MU)}:
It corresponds to the maximum memory consumption of the running container under test. Memory usage is measured in bytes.

-\textit{Execution Time (T)}:
Program execution time is measured in seconds.

We recall that our tool is able to evaluate other non-functional properties of generated code such as code generation time, compilation time, code size, CPU usage.
 
\subsubsection{Setting up Infrastructure}
\begin{figure}[h]
	\centering
	\includegraphics[width=1\linewidth]{Ressources/settingup.pdf}
	\caption{Comparison of average memory consumption and execution time of FFmpeg containers compiled with standard GCC optimization options}
\end{figure}
To assess our approach, we configure our previously proposed container-based infrastructure for non-functional testing of code generators in order to run experiments on the Haxe case study.
Figure 3 shows a big picture of the testing and monitoring infrastructure considered in these experiments.


\begin{table}[h]
	\centering

	\resizebox{\columnwidth}{!}{%
	\begin{tabular}{|l|l|l|l|l|l|l|}
		\hline
		\textbf{Benchmark}                 & \textbf{TestSuite} & \textbf{Std\_dev}                & \textbf{TestSuite} & \textbf{Std\_dev}             & \textbf{TestSuite} & \textbf{Std\_dev}              \\ \hline
		& \textbf{TS1}       & 0,55                             & \textbf{TS8}       & 0,24                          & \textbf{TS15}      & 0,73                           \\ \cline{2-7} 
		& \textbf{TS2}       & 0,29                             & \textbf{TS9}       & 0,22                          & \textbf{TS16}      & 0,12                           \\ \cline{2-7} 
		& \textbf{TS3}       & 0,34                             & \textbf{TS10}      & 0,10                          & \textbf{TS17}      & 0,31                           \\ \cline{2-7} 
		& \textbf{TS4}       & 2,51                             & \textbf{TS11}      & 0,17                          & \textbf{TS18}      & 0,34                           \\ \cline{2-7} 
		& \textbf{TS5}       & 1,53                             & \textbf{TS12}      & 0,28                          & \textbf{TS19}      & \cellcolor[HTML]{C0C0C0}120,61 \\ \cline{2-7} 
		& \textbf{TS6}       & 43,50                            & \textbf{TS13}      & 0,33                         & \multicolumn{2}{l|}{\multirow{2}{*}{}} \\ \cline{2-5}
		\multirow{-7}{*}{\textbf{Color}}   & \textbf{TS7}       & 0,50                             & \textbf{TS14}      & 1,88                          & \multicolumn{2}{l|}{}                  \\ \hline
		& \textbf{TS1}       & 0,35                             & \textbf{TS18}      & 0,16                          & \textbf{TS35}      & 1,30                           \\ \cline{2-7} 
		& \textbf{TS2}       & 0,07                             & \textbf{TS19}      & 0,60                          & \textbf{TS36}      & 1,13                           \\ \cline{2-7} 
		& \textbf{TS3}       & 0,30                             & \textbf{TS20}      & 5,79                          & \textbf{TS37}      & 2,02                           \\ \cline{2-7} 
		& \textbf{TS4}       & \cellcolor[HTML]{C0C0C0}27299,89 & \textbf{TS21}      & 0,47                          & \textbf{TS38}      & 0,26                           \\ \cline{2-7} 
		& \textbf{TS5}       & 6,12                             & \textbf{TS22}      & 2,74                          & \textbf{TS39}      & 0,16                           \\ \cline{2-7} 
		& \textbf{TS6}       & 21,90                            & \textbf{TS23}      & 2,14                          & \textbf{TS40}      & 8,12                           \\ \cline{2-7} 
		& \textbf{TS7}       & 0,41                             & \textbf{TS24}      & 3,79                          & \textbf{TS41}      & 5,45                           \\ \cline{2-7} 
		& \textbf{TS8}       & 0,28                             & \textbf{TS25}      & 0,19                          & \textbf{TS42}      & 0,11                           \\ \cline{2-7} 
		& \textbf{TS9}       & 0,78                             & \textbf{TS26}      & 0,13                          & \textbf{TS43}      & 1,41                           \\ \cline{2-7} 
		& \textbf{TS10}      & 1,82                             & \textbf{TS27}      & 5,59                          & \textbf{TS44}      & 1,56                           \\ \cline{2-7} 
		& \textbf{TS11}      & \cellcolor[HTML]{C0C0C0}180,68   & \textbf{TS28}      & 1,71                          & \textbf{TS45}      & 0,11                           \\ \cline{2-7} 
		& \textbf{TS12}      & \cellcolor[HTML]{C0C0C0}185,02   & \textbf{TS29}      & 0,26                          & \textbf{TS46}      & 1,04                           \\ \cline{2-7} 
		& \textbf{TS13}      & \cellcolor[HTML]{C0C0C0}128,78   & \textbf{TS30}      & 0,44                          & \textbf{TS47}      & 0,23                           \\ \cline{2-7} 
		& \textbf{TS14}      & 0,71                             & \textbf{TS31}      & 1,71                          & \textbf{TS48}      & 1,34                           \\ \cline{2-7} 
		& \textbf{TS15}      & 0,12                             & \textbf{TS32}      & 2,42                          & \textbf{TS49}      & 1,86                           \\ \cline{2-7} 
		& \textbf{TS16}      & 0,65                             & \textbf{TS33}      & 8,29                          & \textbf{TS50}      & 1,28                           \\ \cline{2-7} 
		\multirow{-17}{*}{\textbf{Core}}   & \textbf{TS17}      & 0,26                             & \textbf{TS34}      & 5,25                          & \textbf{TS51}      & 3,53                           \\ \hline
		& \textbf{TS1}       & 31,65                            & \textbf{TS3}       & 30,34                         & \textbf{TS5}       & 0,40                           \\ \cline{2-7} 
		\multirow{-2}{*}{\textbf{Hxmath}}  & \textbf{TS2}       & 4,27                             & \textbf{TS4}       & 0,25                          & \textbf{TS6}       & 0,87                           \\ \hline
		& \textbf{TS1}       & 0,28                             & \textbf{TS3}       & \cellcolor[HTML]{C0C0C0}95,36 & \textbf{TS4}       & 1,49                           \\ \cline{2-7} 
		\multirow{-2}{*}{\textbf{Format}}  & \textbf{TS2}       & \cellcolor[HTML]{C0C0C0}64,94    & \multicolumn{4}{l|}{\textbf{}}                                                                           \\ \hline
		\textbf{Promise}                   & \textbf{TS1}       & 0,29                             & \textbf{TS2}       & 13,21                         & \textbf{TS3}       & 1,21                           \\ \hline
		& \textbf{TS1}       & 0,13                             & \textbf{TS3}       & 0,13                          & \textbf{TS4}       & 1,40                           \\ \cline{2-7} 
		\multirow{-2}{*}{\textbf{Culture}} & \textbf{TS2}       & 0,10                             & \multicolumn{4}{l|}{}                                                                                    \\ \hline
		\textbf{Math}                      & \textbf{TS1}       & \cellcolor[HTML]{C0C0C0}642,85   & \textbf{TS2}       & 28,32                         & \textbf{TS3}       & 24,40                          \\ \hline
	\end{tabular}%
	}
		\caption{My caption}
		\label{my-label}
\end{table}

% Please add the following required packages to your document preamble:
% \usepackage{multirow}
% Please add the following required packages to your document preamble:
% \usepackage{multirow}
% \usepackage[table,xcdraw]{xcolor}
% If you use beamer only pass "xcolor=table" option, i.e. \documentclass[xcolor=table]{beamer}
\begin{table}[]
	\centering

	\resizebox{\columnwidth}{!}{%
	\begin{tabular}{|l|l|l|l|l|l|l|}
		\hline
		\textbf{Benchmark}                 & \textbf{TestSuite} & \textbf{Std\_dev}               & \textbf{TestSuite} & \textbf{Std\_dev}              & \textbf{TestSuite} & \textbf{Std\_dev}              \\ \hline
		& \textbf{TS1}       & 10,19                           & \textbf{TS8}       & 1,23                           & \textbf{TS15}      & 14,44                          \\ \cline{2-7} 
		& \textbf{TS2}       & 1,17                            & \textbf{TS9}       & 1,95                           & \textbf{TS16}      & 1,13                           \\ \cline{2-7} 
		& \textbf{TS3}       & 0,89                            & \textbf{TS10}      & 1,27                           & \textbf{TS17}      & 0,72                           \\ \cline{2-7} 
		& \textbf{TS4}       & 30,34                           & \textbf{TS11}      & 0,57                           & \textbf{TS18}      & 0,97                           \\ \cline{2-7} 
		& \textbf{TS5}       & 31,79                           & \textbf{TS12}      & 1,11                           & \textbf{TS19}      & \cellcolor[HTML]{C0C0C0}777,32 \\ \cline{2-7} 
		& \textbf{TS6}       & \cellcolor[HTML]{C0C0C0}593,05  & \textbf{TS13}      & 0,46                           & \multicolumn{2}{l|}{}                               \\ \cline{2-5}
		\multirow{-7}{*}{\textbf{Color}}   & \textbf{TS7}       & 12,14                           & \textbf{TS14}      & 45,90                          & \multicolumn{2}{l|}{\multirow{-2}{*}{}}             \\ \hline
		& \textbf{TS1}       & 1,40                            & \textbf{TS18}      & 1,00                           & \textbf{TS35}      & 14,13                          \\ \cline{2-7} 
		& \textbf{TS2}       & 1,17                            & \textbf{TS19}      & 20,37                          & \textbf{TS36}      & 32,41                          \\ \cline{2-7} 
		& \textbf{TS3}       & 0,60                            & \textbf{TS20}      & 128,23                         & \textbf{TS37}      & 22,72                          \\ \cline{2-7} 
		& \textbf{TS4}       & \cellcolor[HTML]{C0C0C0}403,15  & \textbf{TS21}      & 24,38                          & \textbf{TS38}      & 2,19                           \\ \cline{2-7} 
		& \textbf{TS5}       & 41,95                           & \textbf{TS22}      & 76,24                          & \textbf{TS39}      & 0,26                           \\ \cline{2-7} 
		& \textbf{TS6}       & 203,55                          & \textbf{TS23}      & 18,82                          & \textbf{TS40}      & 126,29                         \\ \cline{2-7} 
		& \textbf{TS7}       & 19,69                           & \textbf{TS24}      & 72,01                          & \textbf{TS41}      & 31,01                          \\ \cline{2-7} 
		& \textbf{TS8}       & 0,78                            & \textbf{TS25}      & 0,21                           & \textbf{TS42}      & 0,93                           \\ \cline{2-7} 
		& \textbf{TS9}       & 30,41                           & \textbf{TS26}      & 2,30                           & \textbf{TS43}      & 50,36                          \\ \cline{2-7} 
		& \textbf{TS10}      & 57,19                           & \textbf{TS27}      & 101,53                         & \textbf{TS44}      & 12,56                          \\ \cline{2-7} 
		& \textbf{TS11}      & 68,92                           & \textbf{TS28}      & 43,67                          & \textbf{TS45}      & 0,91                           \\ \cline{2-7} 
		& \textbf{TS12}      & 74,19                           & TS29               & 0,90                           & \textbf{TS46}      & 27,28                          \\ \cline{2-7} 
		& \textbf{TS13}      & 263,99                          & \textbf{TS30}      & 4,02                           & \textbf{TS47}      & 1,10                           \\ \cline{2-7} 
		& \textbf{TS14}      & 19,89                           & \textbf{TS31}      & 52,35                          & \textbf{TS48}      & 15,40                          \\ \cline{2-7} 
		& \textbf{TS15}      & 0,30                            & \textbf{TS32}      & 134,75                         & \textbf{TS49}      & 37,01                          \\ \cline{2-7} 
		& \textbf{TS16}      & 28,29                           & \textbf{TS33}      & 82,66                          & \textbf{TS50}      & 23,29                          \\ \cline{2-7} 
		\multirow{-17}{*}{\textbf{Core}}            & \textbf{TS17}      & 1,16                            & \textbf{TS34}      & 89,57                          & \textbf{TS51}      & 1,28                           \\ \hline
		& \textbf{TS1}       & \cellcolor[HTML]{C0C0C0}444,18  & \textbf{TS3}       & \cellcolor[HTML]{C0C0C0}425,65 & \textbf{TS5}       & 17,69                          \\ \cline{2-7} 
		\multirow{-2}{*}{\textbf{Hxmath}}  & \textbf{TS2}       & 154,80                          & \textbf{TS4}       & 0,96                           & \textbf{TS6}       & 46,13                          \\ \hline
		& \textbf{TS1}       & 0,74                            & \textbf{TS3}       & 255,36                         & \textbf{TS4}       & 8,40                           \\ \cline{2-7} 
		\multirow{-2}{*}{\textbf{Format}}  & \textbf{TS2}       & 106,87                          & \multicolumn{4}{l|}{\textbf{}}                                                                            \\ \hline
		\textbf{Promise}                   & \textbf{TS1}       & 0,30                            & \textbf{TS2}       & 58,76                          & \textbf{TS3}       & 20,04                          \\ \hline
		& \textbf{TS1}       & 1,28                            & \textbf{TS3}       & 0,58                           & \textbf{TS4}       & 15,69                          \\ \cline{2-7} 
		\multirow{-2}{*}{\textbf{Culture}} & \textbf{TS2}       & 4,51                            & \multicolumn{4}{l|}{}                                                                                     \\ \hline
		\textbf{Math}                      & \textbf{TS1}       & \cellcolor[HTML]{C0C0C0}1041,53 & \textbf{TS2}       & 234,93                         & \textbf{TS3}       & 281,12                         \\ \hline
	\end{tabular}%
}
	\caption{My caption}
	\label{my-label}
\end{table}


\begin{table*}[h]
	\centering

	\resizebox{\linewidth}{!}{%
	\begin{tabular}{|l|l|l|l|l|l|l|l|l|l|l|}
		\hline
		\multirow{2}{*}{}    & \multicolumn{2}{c|}{\textbf{JS}}      & \multicolumn{2}{c|}{\textbf{JAVA}}    & \multicolumn{2}{c|}{\textbf{C++}}     & \multicolumn{2}{c|}{\textbf{CS}}      & \multicolumn{2}{c|}{\textbf{PHP}}     \\ \cline{2-11} 
		& \textbf{Memory} & \textbf{Factor} & \textbf{Memory} & \textbf{Factor} & \textbf{Memory} & \textbf{Factor} & \textbf{Memory} & \textbf{Factor} & \textbf{Memory} & \textbf{Factor} \\ \hline
		\textbf{Color\_TS6}  & 900,70              & x1.0            & 1362,55              & x1,5            & 2275,49             & x2,5            & 1283,31             & x1,4            & 758,79              & x0,8            \\ \hline
		\textbf{Color\_TS19} & 253,01              & x1.0            & 819,92              & x3,2            & 923,99              & x3,7            & 327,61              & x1,3            & 2189,86             & x8,7            \\ \hline
		\textbf{Core\_TS4}   & 303,09              & x1.0            & 768,22              & x2,5            & 618,42              & x2              & 235,75              & x0,8            & 1237,15             & x4,1            \\ \hline
		\textbf{Hxmath\_TS1} & 104,00              & x1.0            & 335,50              & x3,2            & 296,43              & x2,9            & 156,41              & x1,5            & 1192,98             & x11,5           \\ \hline
		\textbf{Hxmath\_TS3} & 111,68              & x1.0            & 389,73              & x3,5            & 273,12              & x2,4            & 136,49              & x1,2            & 1146,05             & x10,3           \\ \hline
		\textbf{Math\_TS1}   & 493,66              & x1.0            & 831,44              & x1,7            & 1492,97             & x3              & 806,33              & x1,6            & 3088,15             & x6,3            \\ \hline
	\end{tabular}%
}
	\caption{My caption}
	\label{my-label}
\end{table*}


\begin{table*}[h]
	\centering

	\resizebox{\linewidth}{!}{%
	\begin{tabular}{|l|l|l|l|l|l|l|l|l|l|l|}
		\hline
		\multirow{2}{*}{}    & \multicolumn{2}{c|}{\textbf{JS}}   & \multicolumn{2}{c|}{\textbf{JAVA}} & \multicolumn{2}{c|}{\textbf{C++}}  & \multicolumn{2}{c|}{\textbf{CS}}   & \multicolumn{2}{c|}{\textbf{PHP}}  \\ \cline{2-11} 
		& \textbf{Time} & \textbf{Factor} & \textbf{Time} & \textbf{Factor} & \textbf{Time} & \textbf{Factor} & \textbf{Time} & \textbf{Factor} & \textbf{Time} & \textbf{Factor} \\ \hline
		\textbf{Color\_TS19} & 4,52             & x1.0            & 8,61             & x1,9            & 10,73            & x2,4            & 14,99            & x3,3            & 279,27           & x61,8           \\ \hline
		\textbf{Core\_TS4}   & 665,78           & x1.0            & 416,85           & x0,6            & 699,11           & x1,1            & 1161,29          & x1,7            & 61777,21         & x92,8           \\ \hline
		\textbf{Core\_TS11}  & 4,27             & x1.0            & 1,80             & x0,4            & 1,57             & x0,4            & 5,71             & x1,3            & 407,33           & x95,4           \\ \hline
		\textbf{Core\_TS12}  & 4,71             & x1.0            & 2,06             & x0,4            & 1,60             & x0,3            & 5,36             & x1,1            & 417,14           & x88,6           \\ \hline
		\textbf{Core\_TS13}  & 6,26             & x1.0            & 5,91             & x0,9            & 11,04            & x1,8            & 14,14            & x2,3            & 297,21           & x47,5           \\ \hline
		\textbf{Format\_TS2}   & 2,31             & x1.0            & 2,10             & x0,9            & 1,81             & x0,8            & 6,08             & x2,6            & 148,24           & x64,1           \\ \hline
		\textbf{Format\_TS3}   & 5,40             & x1.0            & 5,03             & x0,9            & 7,67             & x1,4            & 12,38            & x2,3            & 220,76           & x40,9           \\ \hline
		\textbf{Math\_TS1}   & 3,01             & x1.0            & 12,51            & x4,2            & 16,30            & x5,4            & 14,14            & x4,7            & 1448,90          & x481,7          \\ \hline
	\end{tabular}%
}
	\caption{My caption}
	\label{my-label}
\end{table*}



First, we create a new Docker image in where we install the Haxe code generators and compilers (through the configuration file "Dockerfile"). Then a new instance of that image is created. It takes as an input the Haxe library we would to test and the list of test suites (step 1). It produces as an output the source code and binaries that have to be executed. These files are saved in a shared repository.
In Docker environment, this repository is called Data Volume. A data volume is a specially-designated directory within containers that share data with the host machine. So, when
we execute the generated test suites, we provide a shared volume with
the host machine so that, binaries can be executed in the execution container (Step 2). In fact, for the code execution we created, as well, a new Docker image in where we install all execution tools and environments such as php interpreter, NodeJS, etc. 

In the meantime, while running test suites inside the container, we collect runtime resource usage data using cAdvisor (step 3). The cAdvisor Docker image does not need any configuration on the host machine. We have just to run it on our host machine. It will then have access to resource usage and performance characteristics of all running containers. This image uses the cgroups mechanism described previously to collect, aggregate, process, and export ephemeral real-time information about running containers. Then, it reports all statistics via web UI (\textit{http://localhost:8080}) to view live resource consumption of each container. cAdvisor has been widely used in different projects such as Heapster\footnote{\url{https://github.com/kubernetes/heapster}} and Google Cloud Platform\footnote{\url{https://cloud.google.com/}}. In this experiment, we choose to gather information about the memory usage of running container.
Afterwards, we record these data into a new time-series database using our InfluxDB back-end container (step 4). Thus, we define its corresponding ip port into the monitoring component so that, container statistics are sent over TCP port (e.g., \textit{8083}) exposed by the database component. 

Next, we run Grafana and we link it to InfluxDB by setting up the data source port 8086 so that, it can easily request data from the database. We recall that InfluxDB also provides a web UI to query the database and show graphs (step 5). But, Grafana let us display live results over time in much pretty looking graphs. Same as InfluxDB, we use SQL queries to extract non-functional metrics from the database for visualization and analysis (step 6). In our experiment, we are gathering the maximum memory usage values without presenting the graphs of resource usage profiles.

To obtain comparable and reproducible results, we use the same hardware across all experiments: an AMD A10-7700K APU Radeon(TM) R7 Graphics processor with 4 CPU cores (2.0 GHz), running Linux with a 64 bit kernel and 16 GB of system memory.

\subsection{Experimental Results}

We now conduct experiments based on the Haxe benchmark. We run each test suite 1K times and we report the execution time and memory usage across the different target languages: Java, JS, C++, CS, and PHP. 
The goal of running these experiments is to observe and compare the behavior of generated code regarding the testing load. We recall, as mentioned in the motivation, that we are not using any oracle function to detect inconsistencies. However, we rely on the comparison results across different targets to define code generator inconsistencies.
Thus, we use, as a quality metric, the standard deviation to quantify the amount of variation among execution traces (i.e., memory usage or execution time) and that for the five target languages. A low standard deviation of a test suite execution, indicates that the data points (execution time or memory usage data) tend to be close to the mean which we consider as an acceptable behavior.  
On the other hand, a high standard deviation indicates that one or more data points are spread out over a wider range of values which can be more likely interpreted as a code generator inconsistency. 


In Table 3, we report the comparison results of running the benchmark in terms of execution speed. At the first glance, we can clearly see that all standard deviations are more mostly close to 0 - 8 interval. It is completely normal to get such small deviations, because we are comparing the execution time of test suites that are written in heterogeneous languages and executed using different technologies (e.g., interpreters for PHP, JVM for JAVA, etc.). So, it is expected to get a small deviation between the execution times after running the test suite in different languages. However, we remark in the same table, that there are some variation points where the deviation is relatively high. We count 8 test suites where the deviation is higher than 60 (highlighted in gray). We choose this value (i.e., standard deviation = 60) as a threshold to designate the points where the variation is extremely high. Thus, we consider values higher than 60 as a potential possibility that a non-functional bug can occur. These variations can be explained by the fact that the execution speed of one or more test suites varies considerably from one language to another. This argues the idea that the code generator has produced a suspect behavior of code for one or more target language. We provide later better explanation in order to detect the faulty code generators.

Similarly, table 4 resumes the comparison results of test suites execution regarding memory usage. The variation in this experiment are more important than previous results. This can be argued by the fact that the memory utilization and allocation patterns are different for each language. Nevertheless, we can recognize some points where the variation is extremely high. Thus, we choose a threshold value equal to 400 and we highlighted, in gray, these extreme points. Thus, we detected 6 test suites where the the variation is extremely high. 
One of the reasons that caused this variation may occur when the test suite executes some parts of the code (in a specific language) that are so greedy in terms of resources. This may be not the case when the variation is lower than 10 for example.
We assume then that the faulty code generator, in this case, represents a threat for software quality since it can generate a code that is very resource consuming.
 
The inconsistencies we are trying to find here are more related to the incorrect memory utilization patterns produced by the faulty code generator. Such inconsistencies may come from an inadequate type usage, high resource instantiation, etc.

Now that we have observed the non-functional behavior of test suites execution in different languages, we can analyze the extreme points we have detected in previous tables to dig more in deep the source of such deviation.
For that reason, we present in table 5 and 6 the raw data values of these extreme test suites in terms of execution time and memory usage. 

Table 5 is showing the execution time of each test suite in a specific target language. We provides also factors of execution times among test suites running in different languages by taking as a baseline the JS version. 
Prior to this, wa calculated the average execution time and memory usage for all test suites per language (for all benchmark libraries). We found that JS has the lowest memory usage and execution time therefore, we choose it as a baseline. Based on these results, we can clearly see that the PHP code has the lowest performance with a factor ranging from x40.9 for testsuite 3 in benchamrk Format (Format\_TS3) to x481.7 for Math\_TS1. We remark also that running Core\_TS4 takes 61777 seconds (almost 17 hours) compared to a 416 seconds (around 6 minutes) in JAVA which is a very large gap. The highest factor detected for other languages ranges from 0.3 to 5.4 which is not negligible but it represents a small deviation compared to PHP version. While it is true that we are comparing different versions of generated code, it was expected to get some variations while running test cases in terms of execution time. However, in the case of PHP code generator it is far to be a simple variation but it is more likely to be a non-functional bug.  
 


\subsection{Discussions}








