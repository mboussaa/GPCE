%\section{Background and Motivation}

%\vspace*{-0.3cm}


\section{Motivation}

\subsection{Code Generator Families}
This paper is based on the intuition that a code generator is often a member of a family of code generators.
\begin{mydef}[\textbf{Code generator family}]
We define a code generator family as a set of code generators that takes as input the same language/model and generate code for different target platforms.
\end{mydef}
%Thus, we can automatically compare the performance between different versions
%of generated code (coming from the same source program). Based on this comparison, we can automatically detect singular resource consumptions that could reveal a code generator bug.
The availability of multiple generators with comparable functionality enables to apply the idea of differential testing~\cite{mckeeman1998differential} to detect code generator issues.
As motivating examples for this research, we can cite three approaches that intensively develop and use code generator families: 
\paragraph{a. Haxe.} 	Haxe\footnote{\url{http://haxe.org/}}~\cite{dasnois2011haxe} is an open source toolkit for cross-platform development which compiles to a number of different programming platforms, including JavaScript, Flash, PHP, C++, C\#, and Java. Haxe involves many features: the Haxe language, multi-platform compilers, and different native libraries. The Haxe language is a high-level programming language which is strictly typed. This language supports both, functional and object-oriented programming paradigms. It has a common type hierarchy, making certain API available on every target platform. Moreover, Haxe comes with a set of code generators that translate manually-written code (in Haxe language) to different target languages and platforms.  
%Haxe code can be compiled for applications running on desktop, mobile, and web platforms. Compilers ensure the correctness of user code in terms of syntax and type safety. Haxe comes also with a set of standard libraries that can be used on all supported targets and platform-specific libraries for each of them. One of the main uses of Haxe is to develop cross-platform games or cross-platform libraries that can run on mobile, on the Web or on a Desktop.  
This project is popular (more than \num{1440} stars on GitHub).

\paragraph{b. ThingML.} ThingML\footnote{\url{http://thingml.org/}} is a modeling language for embedded and distributed systems~\cite{fleurey2011mde}. The idea of ThingML is to develop a practical model-driven software engineering tool-chain which targets resource-constrained embedded systems such as low-power sensors and microcontroller-based devices. ThingML is developed as a domain-specific modeling language which includes concepts to describe both software components and communication protocols. The formalism used is a combination of architecture models, state machines and an imperative action language. The ThingML tool-set provides a  code generator families  to translate ThingML to C, Java and JavaScript. It includes a set of variants for the C and JavaScript code generators to target different embedded systems and their constraints. 
This project is still confidential, but it is a good candidate to represent the modeling community practices.
 
\paragraph{c. TypeScript.} TypeScript\footnote{\url{https://www.typescriptlang.org/}}is a typed superset of JavaScript that compiles to plain JavaScript~\cite{rastogi2015safe}. In fact, it does not compile to only one version of JavaScript. It can transform TypeScript to EcmaScript 3, 5 or 6. It can generate JavaScript that uses different system modules ('none', 'commonjs', 'amd', 'system', 'umd', 'es6', or 'es2015')\footnote{Each of this variation point can target different code generators (function \textit{emitES6Module} vs \textit{emitUMDModule} in emitter.ts for example).}. 
This project is popular (more than \num{12619} stars on GitHub).


\subsection{Functional Correctness of a Code Generator Family}

A reliable and acceptable way to increase the confidence in the correctness of a code generator family is to validate and check the functionality of generated code, which is a common practice for compiler validation and testing~\cite{jorges2014back,stuermer2007systematic,sturmer2005overview}.
Therefore, developers try to check the syntactic and semantic correctness of the generated code by means of different techniques such as static analysis, test suites, etc., and ensure that the code is behaving correctly.  
Functionally testing a code generator family in this case would be simple. Since the generated programs have the same input program, the oracle can be defined as the comparison between the functional outputs of these programs which should be the same.
%In model-based testing for example~\cite{jorges2014back,stuermer2007systematic}, testing code generators focuses on testing the generated code against its design. Thus, the model and the generated code are executed in parallel, by means of simulations, with the same set of test suites. Afterwards, the two outputs are compared with respect to certain acceptance criteria. Test cases, in this case, can be designed to maximize the code or model coverage~\cite{sturmer2005overview}. 

Based on the three sample projects presented above, we remark that all GitHub code repositories of the corresponding projects use unit tests to check the correctness of code generators.  

\subsection{Performance Evaluation of a Code Generator Family}

Another important aspect of code generators testing is to test the non-functional properties of produced code. 
Code generators have to respect different requirements to preserve software reliability and quality~\cite{demertzi2011analyzing}. 
In this case, ensuring the code quality of generated code can require examining several non-functional properties such as code size, resource or energy consumption, execution time, etc~\cite{pan2006fast}.
A non-efficient code generator might generate defective software artifacts (code smells) that violates common software engineering practices. 
Thus, poor-quality code can affect system reliability and performance (e.g., high resource usage, high execution time, etc.).
Proving that the generated code is functionally correct is not enough to claim the effectiveness of the code generator under test. In looking at the three motivating examples, we can observe that ThingML and TypeScript do not provide any specific tests to check the consistency of code generators regarding the memory or CPU usage properties. Haxe provides two test cases\footnote{https://github.com/HaxeFoundation/haxe/tree/development/tests/benchs} to benchmark the resulting generated code. One serves to benchmark an example in which object allocations are deliberately (over) used to measure how memory access/GC mixes with numeric processing in different target languages. The second test evaluates the network speed across different target platforms.

\subsection{Performance Issues of a Code Generator Family}

%However, we believe that we can detect inefficient code generator, based on existing benchmarks among technical environments comparing the behavior of generated code of a large set  of programs~\cite{hundt2011loop}. 
The potential issues that can reveal an inefficient code generator can be resumed as following:
\begin{itemize}
	\setlength\itemsep{0em}
	\item  the lack of use of a \textbf{specific function that exists in the standard library} of the target language  that can speed or reduce the memory consumption of the resulting program.
	\item the lack of use of \textbf{a specific type that exists in the standard library} of the target language  that can speed or reduce the memory consumption of the resulting program.
	\item  the lack of use of\textbf{ a specific language feature in a target language}  that can speed or reduce the memory consumption of the resulting program. 
\end{itemize}

%Following definition 1, we can automatically compare the  performance between different versions of generated code (coming from the same source program). Based on this comparison, we can automatically detect singular resource consumptions that could reveal a code generator bug. As a result,

The main difficulties with testing the non-functional properties of code generators is that we cannot just observe the execution of produced code, but we have to observe and compare the execution of generated programs with equivalent implementations (i.e., in other languages). Even if there is no explicit oracle to detect inconsistencies, we could benefit from the family of code generators to compare the behavior of several generated programs and detect singular resource consumption profiles that could reveal a code generator bug~\cite{hundt2011loop}. 
Our approach is a black-box testing technique and it does not provide detailed information about the source of the issues, as described above. Nevertheless, we rather provide a mechanism to detect these potential issues within a set of code generator families so that, these issues may be investigated and fixed afterwards by code generators/software maintainers.

Next section discusses the common process used by developers to automatically test the performance of generated code. We also illustrate how we can benefit from the code generators families to identify suspect singular behaviors.


%In short, then, we believe that testing the non-functional properties of code generators remains challenging and time-consuming task because developers have to analyze and verify code for each target platform using platform-dependent tools which makes the task of maintaining code generators very tedious. The heterogeneity of platforms and the diversity of target software languages increase the need of supporting tools that can evaluate the consistency and coherence of generated code regarding the non-functional properties. This paper describes a new approach, based on micro-services as execution platforms, to automate and ease the non-functional testing of code generators. This runtime monitoring infrastructure provides a fine-grained understanding of resource consumption and analysis of generated code's behavior.



 



