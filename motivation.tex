\section{Motivation}
\subsection{Compilers optimizations}
By the past, researchers have shown that the choice of optimization sequences can have an effect on software performance\cite{almagor2004finding,chen2012deconstructing}. As a consequence, optimizing software programs for performance becomes key objective for industries and software developers because they are often willing to pay additional costs to meet specific performance goals especially for resource-constrained systems.

Universal and predefined sequences (e.g. O1 to O3) may not produce always good performance results and may be highly dependent on the benchmark and the source code they have been tested on\cite{almagor2004finding}\cite{hoste2008cole}.
In addition, most software engineering programmers are
not familiar with compiler optimizations and they don't have
the capability of selecting effective optimization sequences.
Furthermore, each of these optimizations interacts with the
code and in turn with all other optimizations in complicated
ways. A code transformation can create opportunities for other transformations. Similarly, a transformation can eliminate opportunities for other transformations. Thus, interactions are too complex and it is quite difficult for users to predict the effectiveness of optimizations on their source code program

To explore the large optimization space, users have to
evaluate the effect of optimizations and optimization combinations and that for different target platforms. Thus, finding the optimal optimization options for an input source code is challenging, very hard and time-consuming problem. Many approaches have attempted to solve this optimization selection problem using techniques such as genetic algorithms, iterative compilation, etc\cite{hoste2008cole,zhong2009tuning,sandran2012genetic,martins2014exploration}.

%problem


We note also that performing optimizations to source code
may be so expensive at the expense of resource usage and
may induce to compiler bugs or crashes. With the increasing
of resource usage, it is important to evaluate the compiler
behavior. Indeed, in resource-constrained environment, compiler optimizations may lead to memory leaks or execution
bottlenecks. 

Thus, a fine-grained understanding of resource
consumption and analysis of compilers behavior regarding
optimizations becomes necessary to ensure the efficiency of generated code.

\subsection{Example: GCC compiler}
We choose GCC compiler as a motivating example in order to explain how we would study the impact of compiler optimizations using a component-based infrastructure for testing and monitoring.
The compiler optimization level design space is very huge which needs a heuristic to explore the search space of feasible optimizations sequences. In next section, we present a search-based technique called Novelty Search for automatic generation of optimization sequences. 

GCC exposes its various optimizations via a number of flags that can be turned on or off through command-line compiler switches. For example, the GNU Compiler Collection (GCC) 4.8.4 compiler provides a wide range of command-line optimizations that we can enable or disable (it includes more than 150 options for optimization). We count 76 optimization options that are appearing in the default standard optimization levels O1-O3 of GCC (see table 1). 10 options are enabled by default and the remaining options are defined within O1, O2, O3 and Ofast levels.
This results in a huge space with $2^{76}$ possible optimization
combinations.
We would note that in GCC there are some optimization options that are enabled by default. In our approach, we didn't involve these optimizations since they don't affect either the performance or size of generated binaries.

Optimization flags in GCC can be turned off by using "fno-"+flag instead of "f"+flag in the beginning of each optimization. We use this technique to play with compiler switches
So, our goal is to investigate different combinations of optimization sequences using Novelty Search (NS) in order to explore as much as possible the search space. 

\begin{table}
	\centering
	\caption{Compiler optimization options within standard optimization levels}
	\scalebox{0.88}{
		\begin{tabular}[c]{|c|p{3cm}||c|p{3cm}|}
			
			
			\cline{1-4}
			Level & Optimization option & Level & Optimization option  \\
			\hline
			O1 & 
			-fauto-inc-dec \newline
			-fcompare-elim \newline
			-fcprop-registers \newline
			-fdce \newline
			-fdefer-pop \newline
			-fdelayed-branch \newline
			-fdse \newline
			-fguess-branch-probability \newline
			-fif-conversion2 \newline
			-fif-conversion \newline
			-fipa-pure-const \newline
			-fipa-profile \newline
			-fipa-reference\newline 
			-fmerge-constants\newline
			-fsplit-wide-types \newline
			-ftree-bit-ccp \newline
			-ftree-builtin-call-dce \newline
			-ftree-ccp \newline
			-ftree-ch \newline
			-ftree-copyrename \newline
			-ftree-dce \newline
			-ftree-dominator-opts \newline
			-ftree-dse \newline
			-ftree-forwprop \newline
			-ftree-fre \newline
			-ftree-phiprop \newline
			-ftree-slsr \newline
			-ftree-sra \newline
			-ftree-pta \newline
			-ftree-ter \newline
			-funit-at-a-time
			
			&
			\multirow{2}{*}{O2} & \multirow{2}{6cm} {
				-fthread-jumps\newline 
				-falign-functions\newline  
				-falign-jumps \newline
				-falign-loops  \newline
				-falign-labels \newline
				-fcaller-saves \newline
				-fcrossjumping \newline
				-fcse-follow-jumps  \newline
				-fcse-skip-blocks \newline
				-fdelete-null-pointer-checks \newline
				-fdevirtualize \newline
				-fexpensive-optimizations \newline
				-fgcse  \newline
				-fgcse-lm  \newline
				-fhoist-adjacent-loads \newline
				-finline-small-functions \newline
				-findirect-inlining \newline
				-fipa-sra \newline
				-foptimize-sibling-calls \newline
				-fpartial-inlining \newline
				-fpeephole2 \newline
				-fregmove \newline
				-freorder-blocks  \newline
				-freorder-functions \newline
				-frerun-cse-after-loop \newline 
				-fsched-interblock \newline 
				-fsched-spec \newline
				-fschedule-insns  \newline
				-fschedule-insns2 \newline
				-fstrict-aliasing \newline
				-fstrict-overflow \newline
				-ftree-switch-conversion\newline -ftree-tail-merge \newline
				-ftree-pre \newline
				-ftree-vrp
			} \\
			\cline{1-2}
			O3 & 
			-finline-functions \newline
			-funswitch-loops\newline
			-fpredictive-commoning \newline
			-fgcse-after-reload \newline
			-ftree-vectorize \newline
			-fvect-cost-model \newline
			-ftree-partial-pre \newline 
			-fipa-cp-clone  & &  \\
			\cline{1-2}
			Ofast & -ffast-math &   &  \\
			\hline
			
		\end{tabular}
	}
\end{table}