%\section{Background and Motivation}

\section{Motivations}

\subsection{Code generator families example}

In different domain, the use of code generator is a common practices. We can cite three approaches that intensively develop and use code generators. 

\paragraph{a. Haxe.} 	Haxe is an open source toolkit for cross-platform development which compiles to a number of different programming platforms, including JavaScript, Flash, PHP, C++, C\# and Java. Haxe involves many features: the Haxe language, multi-platform compilers, and different native libraries. The Haxe language is a high-level programming language which is strictly typed. This language supports both functional programming and object-oriented programming paradigms. It has a common type hierarchy, making certain API available on every targeted platform. Moreover, Haxe comes with a set of code generators that translate manually-written code (in Haxe language) to different target languages and platforms.  Haxe code can be compiled for applications running on desktop, mobile and web platforms. Compilers ensure the correctness of user code in terms of syntax and type safety. Haxe comes also with a set of standard libraries that can be used on all supported targets and platform-specific libraries for each of them. One of the main usage of Haxe is to develop Cross-Platform Games or Cross-Platform library that can run on mobile, on the Web or on a Desktop.  	This project is popular (more than 1440 stars on github).

\paragraph{b. ThingML.} ThingML is a modeling language for embedded and distributed systems. The idea of ThingML is to develop a practical model-driven software engineering tool-chain which targets resource constrained embedded systems such as low-power sensor and microcontroller based devices. ThingML is developed as a domain-specific modeling language which includes concepts to describe both software components and communication protocols. The formalism used is a combination of architecture models, state machines and an imperative action language. The ThingML toolset provides a  code generators family  to translate ThingML to C, Java and JavaScript. It includes a set of variant for the C and JavaScript code generators to target different embedded system and their constraints. 
This project is still confidential but is representative of the modeling community practices.
 
\paragraph{c. TypeScript.} TypeScript is a typed superset of JavaScript that compiles to plain JavaScript. In fact, it does not only compile to one version of JavaScript. It can transform typeScript to EcmaScript 3, 5 or 6. It can generate javascript that use different system modules ('none', 'commonjs', 'amd', 'system', 'umd', 'es6', or 'es2015'.). \footnote{Each of this variation point can target different code generators (function \textit{emitES6Module} vs \textit{emitUMDModule} in emitter.ts for example).} This project is popular (more than 12,619 stars on github).


\subsection{Functional  correctness of a family of code generators}


Based on these three samples, we can observe on github that all of them use tests to check the correctness of the code generator. In this test, all of them cover the correctness of the language feature.  A reliable and accepted way to increase confidence in the correct functioning of code generators is to validate and check the functionality of generated code, which is common practice for compiler validation and testing.
Therefore, developers try to check the syntactic and semantic correctness of generated code by means of different techniques such as static analysis, test suites, etc., and ensure that the code is behaving correctly.  In model-based testing for example~\cite{jorges2014back,stuermer2007systematic}, testing code generators focuses on testing the generated code against its design. Thus, the model and the generated code are executed in parallel, by means of simulations, with the same set of test suites. Afterwards, the two outputs are compared with respect to certain acceptance criteria. Test cases, in this case, can be designed to maximize the code or model coverage~\cite{sturmer2005overview}.

\subsection{Non-Functional  correctness of a family of code generators}


In fact, code generators have to respect different requirements which preserve software reliability and quality~\cite{demertzi2011analyzing}. A non-efficient code generator might generate defective software artifacts (code smells) that violates common software engineering practices. Thus, poor-quality code can affect system reliability and performance (e.g., high resource usage, low execution speed, etc.). Thus, another important aspect of code generator's testing is to test the non-functional properties of produced code. Proving that the generated code is functionally correct is not enough to claim the effectiveness of the code generator under test. In looking at the three motivating example, ThingML and TypeScript does not provide any specific test to check the consistency of the different memory usage or CPU consumption regarding the different code generators and their variants. Haxe provides two test cases \footnote{https://github.com/HaxeFoundation/haxe/tree/development/tests/benchs} to benchmark the resulting generated code. One serves to benchmark an example in which object allocations are deliberately (over) used to measure how memory access/GC mixes with numeric processing in the different target language. The second test mainly bench the network speed for each targeted platform. 

However, based on existing benchmarks between technical environments~\cite{hundt2011loop}, in comparing the behavior of the generated code of a large data set of programs, our feeling is that we could detect inefficient code generators. The kinds of error that we track are the following:
\begin{itemize}
	\item  the lack of use of a \textbf{specific function that exists in the standard library} of the targeted language  that can speed or reduce the memory consumption of the resulting program.
	\item the lack of use of \textbf{a specific type that exists in the standard library} of the targeted language  that can speed or reduce the memory consumption of the resulting program.
	\item  the lack of use of\textbf{ a specific language feature in a targeted language}  that can speed or reduce the memory consumption of the resulting program. 
\end{itemize}

The main difficulties is the fact that, for testing the non functional properties of a code generator, we cannot just observe the execution of the code generator but we have to observe and compare the execution of the generated program. Even if there is no exact oracle to detect an inconsistencies, we could benefit from the family of code generators to compare the behavior of several programs generated from the same source that runs atop different technical stacks. 

Next section discusses the common process used by a developer to automatically test the performance of a generated code and illustrate how we can benefit from the code generators families to identify singular behavior. 


%In short, then, we believe that testing the non-functional properties of code generators remains challenging and time-consuming task because developers have to analyze and verify code for each target platform using platform-dependent tools which makes the task of maintaining code generators very tedious. The heterogeneity of platforms and the diversity of target software languages increase the need of supporting tools that can evaluate the consistency and coherence of generated code regarding the non-functional properties. This paper describes a new approach, based on micro-services as execution platforms, to automate and ease the non-functional testing of code generators. This runtime monitoring infrastructure provides a fine-grained understanding of resource consumption and analysis of generated code's behavior.



 



