\section{Related Work}
Our work is related to iterative compilation research field.
The basic idea of iterative compilation is to explore the compiler optimization space by measuring the impact of optimizations on software performance.
Several research efforts have investigated this optimization problem to catch relevant optimizations regrading performance, energy or code size improvements over standard optimization sequences~\cite{almagor2004finding,hoste2008cole,pan2006fast,zhong2009tuning,pallister2015identifying,chen2012deconstructing,sandran2012genetic,martins2014exploration,fursin2008milepost,lin2008automatic,schulte2014post}. 
The vast majority of the work on iterative compilation focuses on increasing the speedup of new optimized code compared to standard optimizations. 
It has been proven that optimizations are highly dependent on target platform and input program. Compared to our proposal, we rather focus on comparing the resource consumption of optimized code.

Novelty Search has never been applied in the filed of iterative compilation. Our work presents the first attempt to introduce diversity in optimization sequences generation. The idea of NS has been introduced by Lehman et al.~\cite{lehman2008exploiting}. It has been often evaluated in deceptive tasks and especially applied to evolutionary robotics~\cite{risi2010evolving,krvcah2012solving} (in the context of neuroevolution). 
NS can easily be adapted to different research fields. In a previous work~\cite{boussaa2015novelty}, NS has been adapted for test data generation where novelty score was calculated as the Manhattan distance between the different vectors representing test data.
In our NS adaptation, we are measuring the novelty score using the systematic difference between optimization sequences of GCC.

For code generators testing, Stuermer et al.~\cite{stuermer2007systematic} present a systematic test approach for model-based code generators. They investigate the impact of optimization rules for model-based code generation by comparing the output of the code execution with the output of the model execution. 
If these outputs are equivalent, it is assumed that the code generator works as expected. 
They evaluate the effectiveness of this approach by means of testing optimizations performed by the TargetLink code generator. 
They have used Simulink as a simulation environment of models. 
In our approach, we provide a component-based infrastructure to compare non-functional properties of generated code rather than functional ones. 

