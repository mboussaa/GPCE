\begin{abstract}
	The intensive use of generative programming techniques provides an elegant engineering solution to deal with the heterogeneity of platforms or technological stacks. The use of Domain Specifics Language, for example, leads to the creation of numerous code generators and compilers that will automatically translate high-level system
	specifications into multi-target executable code. 
	%These practices reduce clearly the development and maintenance effort by developing at a higher-level of abstraction. 	
	Producing correct and efficient code generator is complex and error-prone. Although software designers provide generally high-level test suites to verify the functional outcome of generated code, it remains challenging and tedious to verify the behavior of produced code in terms of non-functional properties.
	%However, industrial code generators may have a huge number of bugs and 
	This paper describes a practical approach based on a runtime monitoring infrastructure to automatically check potential inefficient code generator. This infrastructure based on  system containers as execution platforms allows a code-generator developer to evaluate the consistency and coherence of generated code regarding the non-functional properties. This approach provides a fine-grained understanding of resource consumption and analysis of components behavior. 
	We evaluate our approach by analyzing the non-functional properties of HAXE, a popular high-level programming language that involves a set of cross-platform code generators able to compile to different target platforms. Our experimental results show that our approach is able to detect some non-functional inconsistencies within HAXE code generators.
	%We evaluate the effectiveness of our approach by verifying the optimizations performed by the GCC compiler, a widely used compiler in software engineering community. We also present a number of case studies, in which the tool was successfully used.
	
	%system containers as execution platform
	
	
	%they are highly dependent on target platforms
\end{abstract}
 

\category{D.3.4}{Programming Languages}{Processors– compilers, code generation}

% general terms are not compulsory anymore,
% you may leave them out
\terms
Generative Programming, Testing, Components

\keywords
code quality, non-functional properties, code generator, testing