\section{Conclusion and Future Work}

In this paper we have described a new approach for testing and monitoring code generators families using a container-based infrastructure. 
We used a set of micro-services in order to provide a fine-grained understanding of resource consumption. 
To validate the approach, we use the proposed approach on an widely used code generator families: Haxe. This evaluation shows that we could find real issue in the existing code generators. In particular, we show that we could find two kinds of errors: the lack of use of a specific function and abstract type that exist in the standard library of a targeted language  that can reduce the memory/CPU consumption of the resulting program.

As a current work, we currently discuss with the Haxe community to submit a patch with the first discoveries. We are also conducting the same evaluation for ThingML and TypeScript. As a future work, we would like to better understand if it exists a threshold that provides a best precision for detecting performance issue in code generators. 



