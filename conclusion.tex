\section{Conclusion and Future Work}

\iffalse
In this paper we have described a new approach for testing and monitoring code generators using a component-based infrastructure. 
We used a set of microservices in order to provide a fine-grained understanding of resource consumption. 
To validate the approach, we investigated the problem of GCC compiler optimizations through the use of Novelty Search as a search engine.  Then, we studied the impact of optimizations on memory consumption and execution time across two case studies. 
Results showed that our approach is able to automatically find better optimization sequences across different programs.

As a future work, we plan to explore more trade-offs among resource usage metrics \eg, the correlation between CPU consumption and platform architectures. 
We also intend to compare our findings using the novelty search approach to multi-objective evolutionary algorithms. 
Finally, our proposed microservice-based approach for testing can easily be adapted and integrated to new case studies, so we would inspect the behavior of different other code generators and try to find non-functional bugs regarding code generation processes.

 
 
Second, we have presented an automated tool for automatic extraction of non-functional properties of optimized code, called NOTICE. NOTICE applies different heuristics (including Novelty Search) and performs non-functional testing of compilers through the monitoring of generated code in a controlled sand-boxing environment. In fact, NOTICE uses a set of micro-services to provide a fine-grained understanding of optimization effects on resource consumption. 
We evaluated the effectiveness of our approach by verifying the optimizations performed by GCC compiler. 
%Then, we studied the impact of optimizations on memory consumption and execution time across two case studies. 
Results showed that our approach is able to automatically extract information about memory and CPU consumption. We were also able to find better optimization sequences than standard GCC optimization levels.

As a future work, we plan to explore more trade-offs among resource usage metrics \eg, the correlation between CPU consumption and platform architectures. 
We also intend to provide more facilities to NOTICE users in order to test optimizations performed by modern compilers such as Clang, LLVM, etc.
Finally, NOTICE can be easily adapted and integrated to new case studies. As an example, we would inspect the behavior of model-based code generators since different optimizations can be performed to generate code from models~\cite{stuermer2007systematic}. Thus, we aim to use the same approach to find non-functional issues during the code generation process.
\fi


